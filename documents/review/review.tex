\documentclass{article}

\usepackage[ngerman]{babel}
\usepackage[utf8]{inputenc}
\usepackage[T1]{fontenc}
\usepackage{hyperref}
\usepackage{csquotes}

\usepackage[
    backend=biber,
    style=apa,
    sortlocale=de_DE,
    natbib=true,
    url=false,
    doi=false,
    sortcites=true,
    sorting=nyt,
    isbn=false,
    hyperref=true,
    backref=false,
    giveninits=false,
    eprint=false]{biblatex}
\addbibresource{../references/bibliography.bib}

\title{Review des Papers "Diskriminierung durch KI" von Luisa Rudin}
\author{Lucy Wehrli}
\date{\today}

\begin{document}
\maketitle

\abstract{
    Dies ist mein Review von Luisas Arbeit "Diskriminierung durch KI".
}
\section{Einwandfreie Bereiche}
Die Leitfrage ist spannend formuliert und regt zum Weiterlesen an. Es entstehen auch Fragen, denn ich denke, dass den meisten Leuten nicht bekannt ist, dass KI diskriminierned sein kann. Die Titel der Kapitel sind ebenfalls gut formuliert. Die Beschreibung der Algorithmen und von KIs ist gut, kurz, prägnant und richtig. Wie die Diskriminierung entsteht ist ebenfalls sehr klar und verständlich erklärt. Das zweite Bild finde ich sehr passend. Im Abschnitt "Auswirkungen der Diskriminierung" finde ich den Gedanken mit dem Vertrauen verlieren extrem gut und wichtig. Die zwei Beispiele finde ich sehr gut ausgesucht, denn sie zeigen das grosse Ausmass der Diskriminierung. Ebenfalls zeigen sie, dass diese Diskriminierung lange unentdeckt blieb und dass niemand damit gerechnet hatte. Die Gedanken und Sätze sind generell sehr schön weitergesponnen und gut verknüpft. Es wird flüssig von Aspekt zu Aspekt und zu Thema weitergeleitet. Ich finde, dass die gesamte Arbeit kurz, klar, prägnant und gut geschrieben ist. 


\section{Ausbaufähige Bereiche}
Die zwei Bilder sind zwar sehr passend, jedoch wäre es schön, wenn sie kurz erklärt werden. Sie werden im Text zwar indirekt erklärt, aber eine kurze Beschreibung wäre angenehm. Ausserdem wäre es toll, wenn die Abbildung 2 noch auf der vorigen Seite Platz gehabt hätte. Ich wäre sehr froh, wenn es einen Teil hätte, der die Situation in einen Kontext setzt. Wie gross ist das Ausmass? Ist es überall auf der Welt gleich schlimm? Wie gross schätzen Wissenschaflter die noch unentdeckten Ausmasse und die Dunkelziffern? Und ist es überhaut möglich durch die genannten Massnahem die Diskriminierung zu stoppen? Wenn ja, wie viel Aufwand und Zeit würde das brauchen? Ist es möglich die KI "gedanklich" umzuprogrammieren oder müsste man von Vorne beginnen? Dies wären Alles sehr spannende Fragen gewesen, damit man sich die Ausmasse und Möglichkeiten besser vorstellen kann. In der Arbeit habe ich auch sechs Rechtschreibefehler gefunden. Dies ist jedoch nicht schlimm, da die Arbeit ja ziemlich lange ist und kein Autokorrekt vorhanden ist. Da können kleine Tippfehler schon mal vorkommen.



\printbibliography

\end{document}
