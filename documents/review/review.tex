\documentclass{article}

\usepackage[ngerman]{babel}
\usepackage[utf8]{inputenc}
\usepackage[T1]{fontenc}
\usepackage{hyperref}
\usepackage{csquotes}

\usepackage[
    backend=biber,
    style=apa,
    sortlocale=de_DE,
    natbib=true,
    url=false,
    doi=false,
    sortcites=true,
    sorting=nyt,
    isbn=false,
    hyperref=true,
    backref=false,
    giveninits=false,
    eprint=false]{biblatex}
\addbibresource{../references/bibliography.bib}

\title{Review des Papers "Diskriminierung durch KI" von Luisa Rudin \dots}
\author{Lucy Wehrli}
\date{\today}

\begin{document}
\maketitle

\abstract{
    Dies ist mein Review von Luisas Arbeit "Diskriminierung durch KI".
}
\section{}
Die Leitfrage ist spannend formuliert und regt zum Weiterlesen an. Es entstehen auch Fragen, denn ich denke, dass den meisten Leuten nicht bekannt ist, dass KI diskriminierned sein kann. Die Titel der Kapitel sind ebenfalls gut formuliert. Die Beschreibung der Algorithmen und von KIs ist gut, kurz, prägnant und richtig. Wie die Diskriminierung entsteht ist ebenfalls sehr klar und verständlich erklärt. Das zweite Bild finde ich sehr passend. Im Abschnitt "Auswirkungen der Diskriminierung" finde ich den Gedanken mit dem Vertrauen verlieren extrem gut und wichtig. Die zwei Beispiele finde ich sehr gut ausgesucht, denn sie zeigen das grosse Ausmass der Diskriminierung. Ebenfalls zeigen sie, dass diese Diskriminierung lange unentdeckt blieb und dass niemand damit gerechnet hatte. 

\section{Bereiche, in denen man ausbauen kann}

\printbibliography

\end{document}
