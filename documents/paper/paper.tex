\documentclass{report}

\usepackage[ngerman]{babel}
\usepackage[utf8]{inputenc}
\usepackage[T1]{fontenc}
\usepackage{hyperref}
\usepackage{csquotes}
\usepackage[a4paper]{geometry}

\usepackage[
    backend=biber,
    style=apa,
    sortlocale=de_DE,
    natbib=true,
    url=false,
    doi=false,
    sortcites=true,
    sorting=nyt,
    isbn=false,
    hyperref=true,
    backref=false,
    giveninits=false,
    eprint=false]{biblatex}
\addbibresource{../references/bibliography.bib}


\title{Wenn die Grenzen zwischen Menschen und KI verwischen}
\author{Lucy Wehrli}
\date{\today}


\begin{document}

\maketitle

\abstract{
    Was für ethische Folgen hat es, wenn die Grenzen zwischen KI und Menschen verwischen und die Unterscheidung immer schwieriger wird?
}

\tableofcontents

\chapter{}
\section{ Einleitung}

Die Kunstliche Intelligenz oder KI bzw. AI ist nun schon länger ein fester Bestandteil in unserem Alltag. In der Schule arbeiten wird damit, viele Schüler nutzen ChatGPT für die Testvorbereitung und in den Nachrichten hört man ziemlich oft davon. Man sieht auch vermehrt auch Videos oder Audios, die von KI erzeugt worden sind, die man fast nicht mehr von realen Videos und Audios unterscheiden kann. Das bringt natürlich einige Kosequenzen mit sich was Kriminalität, Sicherheit, Glaubwürdigkeit und weiter Dinge betrifft. Es können einfach Videos erstellt werden, die zum Beispiel einen Sohn imitieren, der entführt wurde und nun soll Lösegeld gezahlt werden. Doch in Wahrheit ist der Sohn gar nicht entführt und es ist nur eine Betrugsmasche, um viel Geld zu erhalten. Das stellt ganz Sicher etwas mit der Psyche des Menschen an, wenn man sich bewusst wird, was das Ganze eigentlich bedeutet. Es hat also ganz klare ethische Folgen. Dieses ganze Thema werde ich hier behandeln.




\section{Sind KIs wirklich intelligent?}
KI Tools funktionieren alle ziemlich ähnlich. Ich werde hier mit dem Beispiel von ChatGPT arbeiten. ChatGPT wurde mit einer Unmenge an Daten sozusagen gefüttert und kann nun aus allen Daten, die heraussuchen, die für die gestellte Frage relevant sind.
Hier ist jedoch sehr wichtig zu wissen, dass ChatGPT momentan auf dem Stand von 2021 ist, da es 2021 entwickelt wurde und nur mit den bis dahin vorhanden Daten ausgestattet werden konnte. Das heisst, dass ChatGPT zum Beispiel keine Möglichkeit hat, das Resultat eines Fussballspiels von 2023 zu wissen, da ihm ganz einfach die Daten dazu fehlen. Das wirft natürlich ein paar Fragen auf, denn das ganze wird ja Künstliche Intelligenz genannt. Eine so simple Frage sollte doch eine "intelligente" Maschine beantworten können. Intelligenz ist definiert als "die kognitive bzw. geistige Leisungsfähigkeit speziell im Problemelösen. Ein Computer und somit auch KIs sind sehr gut im Probleme lösen und sie sind auch deutlich schneller als ein Mensch. So gelang es auch dem Programm Deep Blue den Schachweltmeister zu besiegen, weil es schlichtweg eine Unmenge an Möglichkeiten vor jedem Zug berechnete. Hierzu war Garry Kasparov gar nicht fähig. Doch sind KIs wirklich geistig leistungsfähig? Hierzu sagen viele Experten nein, denn KIs können nur durch hohe Rechenkraft intelligent wirken. Es ist aber sicherlich schon einigen passiert, dass sie gedacht haben, dass ChatGPT eine Frage versteht oder erklärt, oder sogar etwas denkt. Dies kann sehr schnell problematisch werden, denn so verwischen die Grenzen zwischen menschlich und künstlich extrem schnell und stark. Es ist ganz klar problematisch wenn menschliche Züge wie Denken, Bewusstsein oder sogar Mitgefühl einer KI zugeschrieben werden. 

\section{Turing Test}

Um menschliche und künstliche Zügen trennen zu können, gibt es den sogenannten Turing Test. Dieser ist nach seinem Erfinder Alan Turing benannt. Er wurde im Jahre 1950 das erste Mal angewendet. Er funktioniert wie folgt.
Ein Mensche kommuniziert mit einer KI und mit einem Menschen. Er weiss jedoch nicht wann er mit welchem der zwei Optionen kommuniziert. Wenn der Mensch nach einem gewissen Zeitraum nicht unterscheiden konnte, bei welchem Gegenüber es sich um die KI und bei welchem es sich um den Menschen handelt, gilt die Maschine als intelligent. 2014 wurde diser Test das erste Mal von einer KI bestanden. Das Ergebnis gilt jedoch als umstritten. Es ist jedoch wichtig zu wissen, dass es ChatGPT gelang, den Test zu bestehen. Es ist also laut Turing Test bereits nicht mehr möglich zwischen Menschen und ChatGPT zu unterscheiden. Was hier auch anzumerken ist, ist dass die CAPTCHA Fragen, wo man zum Beispiel gefragt wird, alle Ampeln anzuklicken auch auf dem Turing Test basieren. CAPTCHA bedeutet nämlich "Completely Automated Public Turing-Test to tell Computers and Humans Apart".

\section{ELIZA - Effekt}



\section{Etwas mit Quellen}

Etwas mit Änderung hier am Ende.

Wenn ich eine Quelle zitieren möchte, kann ich das ganze einfach am Ende des Satzes machen \citep{example}. Oder wie \citet{example} sagt, auch mitten im Text.

\printbibliography

\end{document}
